 \documentclass[oneside,a4paper,12pt]{book}
%\pagestyle{headings}
\frontmatter

%=============================================================================

\usepackage{amsthm}
\usepackage{xspace}
\usepackage{float}
\usepackage{ifthen}
\usepackage{amsbsy}
\usepackage{amssymb}
\usepackage{balance}
\usepackage{booktabs}
\usepackage{graphicx}
\usepackage{rotating}
\usepackage{multirow}
\usepackage{needspace}
\usepackage{microtype}
\usepackage{bold-extra}
\usepackage{geometry}
\usepackage{varioref}
\usepackage{xcolor}
\usepackage{textcomp}
\usepackage{listings}
\usepackage[normalem]{ulem} %emphasize still italic
\usepackage{ucs}

% \usepackage[utf8]{inputenc}
\usepackage[htt]{hyphenat}
\usepackage{times}
\usepackage{url}
\usepackage{alltt}
\usepackage{amsmath}
\usepackage{xfrac}
\usepackage{subfigure}
\usepackage{appendix}
\usepackage{stmaryrd}   % for the \shortuparrow
\usepackage[utopia]{quotchap}

\usepackage{setspace}
\usepackage[numbers, sort&compress]{natbib}
\usepackage{mdwlist}        % support for better spaced lists
% allows for temporary adjustment of side margins
\usepackage{chngpage}
\usepackage[normalem]{ulem} 

% lina entered
\usepackage{indentfirst}
\usepackage[parfill]{parskip} %culprit -> fontpage into 2 pages !!!!!!!!!!!!!!!!!!!!!!!!!!!!!!!
\usepackage[bottom]{footmisc}

\usepackage[T1]{fontenc}
\usepackage{caption}
\usepackage{booktabs}
\usepackage{siunitx}
\usepackage{refcount}
\setlength{\parindent}{1.5em}

% constants

\newcounter{qcounter}

% commands
\newcommand{\n}{$\cdot$}
\newcommand{\y}{\checkmark}
\newcommand{\subscript}[1]{$_{\textrm{\footnotesize{#1}}}$}
\newcommand{\superscript}[1]{$^{\textrm{\footnotesize{#1}}}$}
\newcommand{\vertical}[1]{\raisebox{-4em}{\begin{sideways}{#1}\end{sideways}}}
\newcommand\tab[1][1cm]{\hspace*{#1}}

\usepackage{boxedminipage}

\newboolean{showedits}
\setboolean{showedits}{true} % toggle to show or hide edits
\ifthenelse{\boolean{showedits}}
{
       \newcommand{\ugh}[1]{\textcolor{red}{\uwave{#1}}} % please rephrase
       \newcommand{\ins}[1]{\textcolor{blue}{\uline{#1}}} % please insert
       \newcommand{\del}[1]{\textcolor{red}{\sout{#1}}} % please delete
       \newcommand{\chg}[2]{\textcolor{red}{\sout{#1}}{\ra}\textcolor{blue}{\uline{#2}}} % please change
}{
       \newcommand{\ugh}[1]{#1} % please rephrase
       \newcommand{\ins}[1]{#1} % please insert
       \newcommand{\del}[1]{} % please delete
       \newcommand{\chg}[2]{#2}
}


% ============================================================================
% Put edit comments in a really ugly standout display

\usepackage{xcolor}
\usepackage[normalem]{ulem}
\newcommand{\ra}{$\rightarrow$}
\usepackage{amssymb}

% comments \nb{label}{color}{text}
\newboolean{showcomments}
\setboolean{showcomments}{true}
%\setboolean{showcomments}{false}
\ifthenelse{\boolean{showcomments}}
{\newcommand{\nb}[3]{
  {\colorbox{#2}{\bfseries\sffamily\scriptsize\textcolor{white}{#1}}}
  {\textcolor{#2}{\sf\small$\blacktriangleright$\textit{#3}$\blacktriangleleft$}}}
    \newcommand{\version}{\emph{\scriptsize$-$Id$-$}}
%	 \newcommand{\ugh}[1]{\textcolor{red}{\uwave{#1}}} % please rephrase
%	 \newcommand{\ins}[1]{\textcolor{blue}{\uline{#1}}} % please insert
%	 \newcommand{\del}[1]{\textcolor{red}{\sout{#1}}} % please delete
%	 \newcommand{\chg}[2]{\textcolor{red}{\sout{#1}}{\ra}\textcolor{blue}{\uline{#2}}} % please change
	 \newcommand{\chk}[1]{\textcolor{ForestGreen}{#1}} % changed, please check
	}
{\newcommand{\nb}[3]{}
  \newcommand{\version}{}
  \newcommand{\chk}[1]{} % changed, please check
  }
\newcommand\nm[1]{\nb{NM}{violet}{#1}} % add more author macros here
\newcommand\lt[1]{\nb{LT}{orange}{#1}}

\definecolor{green(pigment)}{rgb}{0.0, 0.65, 0.31}
\newcommand\brs[1]{\nb{BS}{green(pigment)}{#1}}
\newcommand\latex[1]{\nb{\LaTeX}{blue}{#1}}
\newcommand\todo[1]{\nb{TO DO}{blue}{#1}}

% ============================================================================
% Make quotes be italic
\renewenvironment{quote}
    {\list{}{\rightmargin\leftmargin}%
     \item\relax\begin{it}}
    {\end{it}\endlist}

\newcommand{\ttimes}{\ensuremath{\times}}

%=============================================================================

\newcommand{\needlines}[1]{\Needspace{#1\baselineskip}}

% source code
\usepackage{xcolor}
\usepackage{textcomp}
\usepackage{listings}
\definecolor{cadmiumred}{rgb}{0.89, 0.0, 0.13}
\definecolor{javablue}{rgb}{0,0,0.6} % for strings
\definecolor{javagreen}{rgb}{0.25,0.5,0.35} % comments
\definecolor{javapurple}{rgb}{0.5,0,0.35} % keywords
\definecolor{javadocblue}{rgb}{0.25,0.35,0.75} % javadoc

\renewcommand{\lstlistingname}{Code}% Listing -> Algorithm
\renewcommand{\lstlistlistingname}{List of \lstlistingname s}% List of Listings -> List of Algorithms

\lstnewenvironment{Java}[1][H]
{\lstset{
	language=Java,
	basicstyle=\footnotesize\ttfamily,
	keywordstyle=\color{javapurple}\bfseries,
	stringstyle=\color{javablue},
	commentstyle=\color{javagreen},
	morecomment=[s][\color{javadocblue}]{/**}{*/},
	numbers=left,
	numberstyle=\tiny\color{black},
	stepnumber=1,
	numbersep=10pt,
	tabsize=4,
	showspaces=false,
	showstringspaces=false,
	breaklines=true,
	captionpos=b,
	%xleftmargin=2em,
	%framexleftmargin=1.5em,
	frame=single,
	float,
	#1
}}
{}

\lstnewenvironment{JVMIS}[1][H]
{\lstset{
	language=JVMIS,
	basicstyle=\footnotesize\ttfamily,
	keywordstyle=\color{javagreen}\bfseries,
	stringstyle=\color{javablue},
	commentstyle=\color{javagreen},
	morecomment=[s][\color{javadocblue}]{/**}{*/},
	numberstyle=\tiny\color{black},
	stepnumber=1,
	numbersep=10pt,
	tabsize=4,
	showspaces=false,
	showstringspaces=false
	breaklines=true,
	captionpos=b,
	frame=single,
	float,
	#1
}}
{}

\renewcommand{\texttt}[1]{%
  \begingroup
  \ttfamily
  \begingroup\lccode`~=`/\lowercase{\endgroup\def~}{/\discretionary{}{}{}}%
  \begingroup\lccode`~=`[\lowercase{\endgroup\def~}{[\discretionary{}{}{}}%
  \begingroup\lccode`~=`.\lowercase{\endgroup\def~}{.\discretionary{}{}{}}%
  \catcode`/=\active\catcode`[=\active\catcode`.=\active
  \scantokens{#1\noexpand}%
  \endgroup
}

\definecolor{codegray}{gray}{0.9}
\newcommand{\code}[1]{
	\colorbox{codegray}
	{\texttt{#1}}
}
\sloppy
%----------------------------------------------------------------------------



%----------------------------------------------------------------------------
% references
\newcommand{\tabref}[1]{\hyperref[{tab:#1}]{Table~\ref*{tab:#1}}}
\newcommand{\figref}[1]{\hyperref[{fig:#1}]{Figure~\ref*{fig:#1}}}
\newcommand{\secref}[1]{\hyperref[{sec:#1}]{Section~\ref*{sec:#1}}}
\newcommand{\subsecref}[1]{\hyperref[{subsec:#1}]{Subsection~\ref*{subsec:#1}}}
\newcommand{\lstref}[1]{\hyperref[{lst:#1}]{Listing~\ref*{lst:#1}}}
\newcommand{\charef}[1]{\hyperref[{ch:#1}]{Chapter~\ref*{ch:#1}}}
\newcommand{\coderef}[1]{\hyperref[{code:#1}]{Code~\ref*{code:#1}}}
\newcommand{\bytecoderef}[1]{\hyperref[{bytecode:#1}]{Bytecode~\ref*{bytecode:#1}}}
\newcommand{\algref}[1]{\hyperref[{alg:#1}]{Algorithm~\ref*{alg:#1}}}
\newcommand{\boxref}[1]{\hyperref[{box:#1}]{Box~\ref*{box:#1}}}

%----------------------------------------------------------------------------

% abbreviations
\tracingcolors 4
\setcounter{tocdepth}{3}
\setcounter{secnumdepth}{3}
\newcommand{\ie}{\emph{i.e.,}\xspace}
\newcommand{\eg}{\emph{e.g.,}\xspace}
\newcommand{\etc}{\emph{etc.}\xspace}
\newcommand{\etal}{\emph{et al.}\xspace}


\newcommand{\newevenside}{
	\ifthenelse{\isodd{\thepage}}{\newpage}{
	\newpage
        \phantom{placeholder} % doesn't appear on page
	\thispagestyle{empty} % if want no header/footer
	\newpage
	}
}

\def\stretchfactor{1}
\newcommand{\mychapter}[1]{\setstretch{1}
    \chapter{#1}\setstretch{\stretchfactor}}

%----------------------------------------------------------------------------
\newcommand{\lessSpace}{\vspace{-1em}}
\DeclareGraphicsExtensions{.pdf,.png}
\graphicspath{{images/}}
\newcommand{\fig}[4]{
	\begin{figure}[#1]
		\centering
		\includegraphics[width=#2\textwidth]{#3}
		\lessSpace
		\caption{\label{fig:#3}#4}
	\end{figure}}

% ===========================================================================

%:CONFIGURE THIS

\newcommand{\thesistitle}{Where does this null come from ?}
\newcommand{\thesisauthor}{Lina Tran}
\newcommand{\thesisauthorOrigin}{Biel/Bienne BE, Switzerland}
\newcommand{\thesisleiter}{Prof.\ Dr.\ Oscar Nierstrasz}
\newcommand{\thesisasst}{	\begin{center}
														Nevena Milojkovi\'{c}\\
														Boris Spasojevi\'{c}
													\end{center}}
\newcommand{\thesisurl}{http://scg.unibe.ch/}
\newcommand{\thesissubtitle}{An Approach to show the exact location where a value was referenced to null}
\newcommand{\thesisdate}{31. July 2016}

% ===========================================================================

\usepackage[ colorlinks=true, urlcolor=black, linkcolor=black,
			citecolor=black, bookmarksnumbered=true, bookmarks=true,
			plainpages=false,
			pdftitle={\thesistitle}, pdfauthor={\thesisauthor},
			pdfsubject={\thesissubtitle}, pdfpagelabels]{hyperref}

\newcommand{\hrref}[2]{\hyperref}
% ===========================================================================
% ===========================================================================

% D O C U M E N T
% % % % % % % % % % % % % % % % % % % % % % % % % % % % % % % % % %
\begin{document}

% T I T L E
% % % % % % % % % % % % % % % % % % % % % % % % % % % % % % % % % %
\begin{titlepage}  
  \begin{center}  
  
  \begin{figure}[t]  
  \vspace*{-2cm}        % to move header logo at the top 
  \center{\includegraphics[scale=0.5]{logos/UNI_Bern.png}}
  \vspace{1in}     
  \end{figure}

    \thispagestyle{empty}
    
    {\bfseries\Huge \thesistitle \par
    \Large \vspace{0.1in} \thesissubtitle \par}

    \vspace{0.3in} 
    \LARGE{\textbf{Bachelor Thesis} \\}
    \vspace{0.4in}

    {\Large \thesisauthor \par from \par \thesisauthorOrigin}
    
    \vspace{0.3in}
    {\Large Faculty of Science \\
            University of Bern \par}
    \vspace{0.3in}
    {\Large \thesisdate \par}
    \vspace{0.3in}
    %Leiter der Arbeit: \par
   {\Large \thesisleiter} \par
      {\Large \thesisasst} \par
   \vspace{0.1in}
    {\Large Software Composition Group \par Institute for Computer Science \par University of Bern, Switzerland \par}
  

  %\vspace{0.5in}
 
 

  \end{center}

\end{titlepage}


% A B S T R A C T
% % % % % % % % % % % % % % % % % % % % % % % % % % % % % % % % % %
\chapter*{\centering Abstract}
\begin{quotation}
\noindent 
A previous study found out that NullPointerExceptions are very serious in Java projects. When a NullPointerExceptions occurs the developer is provided only with a stack trace to where the exception happened. This only gives insight into the effect of the fault but not into its cause. So we have to ask the question when and why this reference was set to null.

The aim of the project is to be able to provide the user with an additional stack trace of where the value was actually set to null, next to the normal stack trace of an exception. We attempt to achieve this goal by instrumenting java source code ideally with a minimal overhead. 

By tracking the null assignments the debugging after a NullPointerException will be simplified. 
\end{quotation}
\clearpage


% C O N T E N T S 
% % % % % % % % % % % % % % % % % % % % % % % % % % % % % % % % % % % % % % % %
\tableofcontents

\mainmatter
%%%%%%%%%%%%%%%%%%%%%%%%%%%%%%%%%%
%%%% NEW CHAPTER %%%%%%%%%%%%%%%%%%%%%
%%%%%%%%%%%%%%%%%%%%%%%%%%%%%%%%%%
\chapter{Introduction}
\label{cha:introduction}
Nowadays, certainly every programmer is confronted with NullPointerExceptions in big Java Projects, whether it is for an enterprise or for private purposes. Not to mention even in small Java Projects they are also heavily present.

So what are those NullPointerExceptions? This thesis is going to attach importance to Java that is a concurrent, class-based, object-oriented programming language. We chose Java because NullPointerExceptions are more serious in this language than in others, e.g. Smalltalk. NullPointerException is a RuntimeException. In Java, an object reference can be assigned with a special null value. The exception is thrown when an application attempts to use an object reference that has the null value. (There are multiple ways this exception can be thrown, like: Calling an instance method on the object referred by a null reference; Accessing or modifying an instance field of the object referred by a null reference and so on.) In Java Projects developers always have to deal with a huge amount of references which means avoiding these NullPointerExceptions is as good as impossible.

On regular meetings among programmers they report what they have been doing and what they are planning to do for the next few weeks. But all too often it is stated that they are trying to fix bugs or have spent a lot of time fixing them. If there would be a way to minimize the time fixing exceptions and allow to work more efficiently, projects would progress much faster.

The main goal of the NullSpy application takes a step to that ideal vision. Anytime developers are facing a NullPointerException they don’t have to spent time on debugging finding where and why a reference was set to null. With NullSpy the exact location of the null assignment is shown next to the ordinary stack trace the Java virtual machine produces.

In this thesis it is explained how the goal mentioned above is achieved step by step, by using a class library Javassist (Java Programming Assistant) which allows us to deal with Java bytecode.

\chapter {Related Work}
In which we learn what have other done to address similar problems. For example, the work of Star \cite{Star89}

\chapter{The Problem}
In which we understand what the problem is in detail.

\chapter {The Solution}
In which you describe your solution.

\chapter {The Validation}
In which you show how well the solution works.

\chapter {Conclusion and Future Work}
In which we step back, have a critical look at the entire work, then conclude, and learn what lays beyond this thesis.

\chapter {Anleitung zu wissenschaftlichen Arbeiten}
This consists of additional documentation, e.g. a tutorial, user guide etc.
Required by the Informatik regulation.

%END Doc
%-------------------------------------------------------

\bibliography{thesis}
\bibliographystyle{plain}

\end{document}